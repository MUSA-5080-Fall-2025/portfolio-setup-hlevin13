% Options for packages loaded elsewhere
% Options for packages loaded elsewhere
\PassOptionsToPackage{unicode}{hyperref}
\PassOptionsToPackage{hyphens}{url}
\PassOptionsToPackage{dvipsnames,svgnames,x11names}{xcolor}
%
\documentclass[
  11pt,
]{article}
\usepackage{xcolor}
\usepackage[margin=1in]{geometry}
\usepackage{amsmath,amssymb}
\setcounter{secnumdepth}{-\maxdimen} % remove section numbering
\usepackage{iftex}
\ifPDFTeX
  \usepackage[T1]{fontenc}
  \usepackage[utf8]{inputenc}
  \usepackage{textcomp} % provide euro and other symbols
\else % if luatex or xetex
  \usepackage{unicode-math} % this also loads fontspec
  \defaultfontfeatures{Scale=MatchLowercase}
  \defaultfontfeatures[\rmfamily]{Ligatures=TeX,Scale=1}
\fi
\usepackage{lmodern}
\ifPDFTeX\else
  % xetex/luatex font selection
\fi
% Use upquote if available, for straight quotes in verbatim environments
\IfFileExists{upquote.sty}{\usepackage{upquote}}{}
\IfFileExists{microtype.sty}{% use microtype if available
  \usepackage[]{microtype}
  \UseMicrotypeSet[protrusion]{basicmath} % disable protrusion for tt fonts
}{}
\makeatletter
\@ifundefined{KOMAClassName}{% if non-KOMA class
  \IfFileExists{parskip.sty}{%
    \usepackage{parskip}
  }{% else
    \setlength{\parindent}{0pt}
    \setlength{\parskip}{6pt plus 2pt minus 1pt}}
}{% if KOMA class
  \KOMAoptions{parskip=half}}
\makeatother
% Make \paragraph and \subparagraph free-standing
\makeatletter
\ifx\paragraph\undefined\else
  \let\oldparagraph\paragraph
  \renewcommand{\paragraph}{
    \@ifstar
      \xxxParagraphStar
      \xxxParagraphNoStar
  }
  \newcommand{\xxxParagraphStar}[1]{\oldparagraph*{#1}\mbox{}}
  \newcommand{\xxxParagraphNoStar}[1]{\oldparagraph{#1}\mbox{}}
\fi
\ifx\subparagraph\undefined\else
  \let\oldsubparagraph\subparagraph
  \renewcommand{\subparagraph}{
    \@ifstar
      \xxxSubParagraphStar
      \xxxSubParagraphNoStar
  }
  \newcommand{\xxxSubParagraphStar}[1]{\oldsubparagraph*{#1}\mbox{}}
  \newcommand{\xxxSubParagraphNoStar}[1]{\oldsubparagraph{#1}\mbox{}}
\fi
\makeatother


\usepackage{longtable,booktabs,array}
\usepackage{calc} % for calculating minipage widths
% Correct order of tables after \paragraph or \subparagraph
\usepackage{etoolbox}
\makeatletter
\patchcmd\longtable{\par}{\if@noskipsec\mbox{}\fi\par}{}{}
\makeatother
% Allow footnotes in longtable head/foot
\IfFileExists{footnotehyper.sty}{\usepackage{footnotehyper}}{\usepackage{footnote}}
\makesavenoteenv{longtable}
\usepackage{graphicx}
\makeatletter
\newsavebox\pandoc@box
\newcommand*\pandocbounded[1]{% scales image to fit in text height/width
  \sbox\pandoc@box{#1}%
  \Gscale@div\@tempa{\textheight}{\dimexpr\ht\pandoc@box+\dp\pandoc@box\relax}%
  \Gscale@div\@tempb{\linewidth}{\wd\pandoc@box}%
  \ifdim\@tempb\p@<\@tempa\p@\let\@tempa\@tempb\fi% select the smaller of both
  \ifdim\@tempa\p@<\p@\scalebox{\@tempa}{\usebox\pandoc@box}%
  \else\usebox{\pandoc@box}%
  \fi%
}
% Set default figure placement to htbp
\def\fps@figure{htbp}
\makeatother





\setlength{\emergencystretch}{3em} % prevent overfull lines

\providecommand{\tightlist}{%
  \setlength{\itemsep}{0pt}\setlength{\parskip}{0pt}}



 


\makeatletter
\@ifpackageloaded{caption}{}{\usepackage{caption}}
\AtBeginDocument{%
\ifdefined\contentsname
  \renewcommand*\contentsname{Table of contents}
\else
  \newcommand\contentsname{Table of contents}
\fi
\ifdefined\listfigurename
  \renewcommand*\listfigurename{List of Figures}
\else
  \newcommand\listfigurename{List of Figures}
\fi
\ifdefined\listtablename
  \renewcommand*\listtablename{List of Tables}
\else
  \newcommand\listtablename{List of Tables}
\fi
\ifdefined\figurename
  \renewcommand*\figurename{Figure}
\else
  \newcommand\figurename{Figure}
\fi
\ifdefined\tablename
  \renewcommand*\tablename{Table}
\else
  \newcommand\tablename{Table}
\fi
}
\@ifpackageloaded{float}{}{\usepackage{float}}
\floatstyle{ruled}
\@ifundefined{c@chapter}{\newfloat{codelisting}{h}{lop}}{\newfloat{codelisting}{h}{lop}[chapter]}
\floatname{codelisting}{Listing}
\newcommand*\listoflistings{\listof{codelisting}{List of Listings}}
\makeatother
\makeatletter
\makeatother
\makeatletter
\@ifpackageloaded{caption}{}{\usepackage{caption}}
\@ifpackageloaded{subcaption}{}{\usepackage{subcaption}}
\makeatother
\usepackage{bookmark}
\IfFileExists{xurl.sty}{\usepackage{xurl}}{} % add URL line breaks if available
\urlstyle{same}
\hypersetup{
  pdftitle={MUSA 5080/CPLN 5920: Public Policy Analytics},
  colorlinks=true,
  linkcolor={blue},
  filecolor={Maroon},
  citecolor={Blue},
  urlcolor={Blue},
  pdfcreator={LaTeX via pandoc}}


\title{MUSA 5080/CPLN 5920: Public Policy Analytics}
\usepackage{etoolbox}
\makeatletter
\providecommand{\subtitle}[1]{% add subtitle to \maketitle
  \apptocmd{\@title}{\par {\large #1 \par}}{}{}
}
\makeatother
\subtitle{Fall 2025}
\author{}
\date{}
\begin{document}
\maketitle


\subsection{Course Information}\label{course-information}

\textbf{Time:} Mondays, 10:15 AM -- 1:14 PM\\
\textbf{Location:} Fagin 214\\
\textbf{Instructor:} Dr.~Elizabeth Delmelle\\
\textbf{Email:}
\href{mailto:delmelle@design.upenn.edu}{\nolinkurl{delmelle@design.upenn.edu}}\\
\textbf{Office Hours:} Mondays 1:30--3:00 PM/ Tuesdays 9:30-10:30am or
by appointment
\href{https://calendly.com/delmelle/open-house-office-hours}{\emph{Use
Link To Sign Up}}\\
\textbf{TA Office Hours:}

\begin{itemize}
\tightlist
\item
  Jiayue Ma (majiayue@design.upenn.edu): Thursdays 10-11am
\item
  Zhanchao Yang (zhanchao@design.upenn.edu): Tuesdays 1-2pm
\end{itemize}

\subsection{Course Description}\label{course-description}

This course teaches advanced spatial analysis and introduces data
science and machine learning tools within the context of urban planning
and public policy. Unlike private-sector data science focused solely on
optimization, our approach emphasizes public goods, governance, and
equity. We'll cover topics including transportation, housing, public
health, and criminal justice, using both spatial tools and predictive
modeling to help guide resource allocation and policy design.

\textbf{Key Focus:} Understanding concepts deeply rather than just
completing code. We emphasize fairness, transparency, and understanding
the implications of our models.

\begin{center}\rule{0.5\linewidth}{0.5pt}\end{center}

\subsection{Learning Outcomes}\label{learning-outcomes}

By the end of the semester, students will be able to:

\begin{itemize}
\tightlist
\item
  Build and evaluate predictive models for public policy questions
\item
  Critically assess model generalizability, effectiveness, and bias
\item
  Navigate the full data science workflow: wrangling, exploration,
  modeling, and communication
\item
  Integrate spatial and temporal variables into policy-oriented models
\item
  Communicate uncertainty, limitations, and equity impacts to
  decision-makers
\item
  Create professional data science portfolios using Quarto
\end{itemize}

\begin{center}\rule{0.5\linewidth}{0.5pt}\end{center}

\subsection{Course Materials}\label{course-materials}

\subsubsection{Required Texts (All Free
Online)}\label{required-texts-all-free-online}

\begin{itemize}
\tightlist
\item
  Ken Steif,
  \href{https://urbanspatial.github.io/PublicPolicyAnalytics/}{\emph{Public
  Policy Analytics}}
\item
  Hadley Wickham et al., \href{https://r4ds.hadley.nz/}{\emph{R for Data
  Science}}
\item
  Robin Lovelace et al.,
  \href{https://r.geocompx.org/}{\emph{Geocomputation with R}}
\item
  Kyle Walker, \href{https://walker-data.com/census-r/}{\emph{Analyzing
  US Census Data}}
\end{itemize}

\subsubsection{Supplemental Material}\label{supplemental-material}

\begin{itemize}
\tightlist
\item
  Selected chapters from
  \href{https://vis4sds.github.io/vis4sds/}{\emph{Visualization for
  Social Data Science}}
\item
  Additional readings provided weekly via Canvas
\end{itemize}

\begin{center}\rule{0.5\linewidth}{0.5pt}\end{center}

\subsection{Assessment Structure}\label{assessment-structure}

\begin{longtable}[]{@{}
  >{\raggedright\arraybackslash}p{(\linewidth - 4\tabcolsep) * \real{0.3333}}
  >{\raggedright\arraybackslash}p{(\linewidth - 4\tabcolsep) * \real{0.2778}}
  >{\raggedright\arraybackslash}p{(\linewidth - 4\tabcolsep) * \real{0.3889}}@{}}
\toprule\noalign{}
\begin{minipage}[b]{\linewidth}\raggedright
Component
\end{minipage} & \begin{minipage}[b]{\linewidth}\raggedright
Weight
\end{minipage} & \begin{minipage}[b]{\linewidth}\raggedright
Description
\end{minipage} \\
\midrule\noalign{}
\endhead
\bottomrule\noalign{}
\endlastfoot
\textbf{Weekly In-Class Quizzes} & 35\% & Concept-focused assessments
(10 quizzes, lowest score dropped) \\
\textbf{Lab Assignments (5 total)} & 20\% & Implementation + feedback
response graded on 3-point scale \\
\textbf{Midterm: House Prediction Competition} & 15\% & Team-based
modeling competition with lightning presentations \\
\textbf{Final Modeling Challenge} & 25\% & Real-world policy problem
requiring model selection and justification \\
\textbf{Participation \& Weekly Notes} & 5\% & Attendance, engagement,
and weekly learning documentation \\
\end{longtable}

\subsubsection{Lab Assignment Structure}\label{lab-assignment-structure}

\textbf{Philosophy:} Lab assignments assess coding implementation,
documentation, and professional response to feedback. Knowledge of
underlying concepts is evaluated through weekly quizzes. Ultimately, the
labs will form your final portfolio for this course so the amount of
effort you put into each assignment is for your own benefit.

\textbf{Lab Assignment Structure:}

Labs 1, 2, 4, and 5 are individual assignments. \textbf{Lab 3 (House
Prediction)} is completed in teams of 3-4 students.

\textbf{Assignment Sequence:}

\begin{enumerate}
\def\labelenumi{\arabic{enumi}.}
\tightlist
\item
  \textbf{Census Data Exploration} (Individual)
\item
  \textbf{Neighborhood Indicators} (Individual)
\item
  \textbf{House Price Prediction Competition} (Team-based) -
  \emph{Serves as Midterm}
\item
  \textbf{Parole Reform Analysis} - Logistic Regression (Individual)
\item
  \textbf{Bike Share Rebalancing} - Space-Time Modeling (Individual)
\end{enumerate}

\textbf{Final Modeling Challenge:} Teams work on a real-world policy
problem and choose the most appropriate modeling approach from the
semester (linear regression, count models, logistic regression, or
space-time modeling). The challenge emphasizes problem framing,
methodology justification, and complete workflow implementation.

\textbf{GitHub-Based Feedback Response:} Each assignment (after the
first) must include a \texttt{feedback-response.md} file addressing:

\begin{itemize}
\tightlist
\item
  How you incorporated previous TA feedback
\item
  Specific improvements made to visual clarity of figures,
  documentation, writeups
\item
  Challenges encountered and solutions attempted
\item
  Questions or areas needing clarification
\end{itemize}

\textbf{Weekly Notes Requirement:} Students maintain a
\texttt{weekly-notes/} folder in their GitHub repo with files named
\texttt{week-XX-notes.md}. Notes should include:

\begin{itemize}
\tightlist
\item
  Key concepts from lecture and readings
\item
  Coding techniques learned and challenges faced
\item
  Questions or confusion points
\item
  Connections to previous weeks or policy applications
\item
  Personal reflections on the material
\end{itemize}

\emph{Notes are checked weekly for completion and effort (not accuracy)
and serve as quiz preparation aids.}

\textbf{Lab Grading Scale:}

\begin{itemize}
\tightlist
\item
  \textbf{2 points:} Complete implementation + feedback incorporation +
  clear documentation.
\item
  \textbf{1 point:} Somewhat complete, poor feedback integration or
  unclear work
\item
  \textbf{0 points:} Not submitted, incomplete, or no evidence of
  engaging with feedback
\end{itemize}

\textbf{Weekly Quiz:} AI advancements have changed how I view
assessment. I am no longer concerned about students ability to complete
coding-based assignments. AI, while certainly not perfect, will
generally produce a solution to assignment prompts. My major concern now
is how well students are able to really comprehend the concepts and
fundamentals of what is being done in order to be appropriate critics of
what AI produces.

Therefore, each class period will begin with an in-person, written quiz
on material from the prior week or the prior lab assignment. There will
be a total of 10 quizzes, but I will drop the lowest one.

\subsection{Course Format}\label{course-format}

\textbf{Structure:} Each 3-hour session combines conceptual lectures
with hands-on labs

\textbf{Expectations:}

\begin{itemize}
\tightlist
\item
  Bring charged laptops for live coding and group work
\item
  Maintain weekly Quarto-based portfolio with reflections and notes
  (\emph{ideally} students will take written notes and then transfer
  these to quarto after class for optimal retainment.)
\item
  Revise past work based on TA feedback for portfolio improvement
\item
  Engage actively in discussions and collaborative problem-solving
\item
  Attend class!
\end{itemize}

\textbf{Technology:} All work will be completed in R using Quarto for
reproducible, professional documentation.

\textbf{GitHub Classroom:} We will use GitHub Classroom for assignment
distribution and submission. Each assignment creates a personal
repository containing starter materials and instructions. Students
customize, complete, and push their work to GitHub. This workflow builds
professional version control skills while enabling efficient feedback
and collaboration.

\begin{center}\rule{0.5\linewidth}{0.5pt}\end{center}

\subsection{Weekly Schedule}\label{weekly-schedule}

\begin{longtable}[]{@{}
  >{\raggedright\arraybackslash}p{(\linewidth - 10\tabcolsep) * \real{0.1667}}
  >{\raggedright\arraybackslash}p{(\linewidth - 10\tabcolsep) * \real{0.1667}}
  >{\raggedright\arraybackslash}p{(\linewidth - 10\tabcolsep) * \real{0.1667}}
  >{\raggedright\arraybackslash}p{(\linewidth - 10\tabcolsep) * \real{0.1667}}
  >{\raggedright\arraybackslash}p{(\linewidth - 10\tabcolsep) * \real{0.1667}}
  >{\raggedright\arraybackslash}p{(\linewidth - 10\tabcolsep) * \real{0.1667}}@{}}
\toprule\noalign{}
\begin{minipage}[b]{\linewidth}\raggedright
Week
\end{minipage} & \begin{minipage}[b]{\linewidth}\raggedright
Date
\end{minipage} & \begin{minipage}[b]{\linewidth}\raggedright
Topic
\end{minipage} & \begin{minipage}[b]{\linewidth}\raggedright
Assessment
\end{minipage} & \begin{minipage}[b]{\linewidth}\raggedright
Lab Assignment
\end{minipage} & \begin{minipage}[b]{\linewidth}\raggedright
GitHub Deliverables
\end{minipage} \\
\midrule\noalign{}
\endhead
\bottomrule\noalign{}
\endlastfoot
1 & Sep 8 & Course Intro • Quarto \& GitHub Setup • R Review & --- &
Setup \& Portfolio Init & Initial repo + Week 1 notes \\
2 & Sep 15 & Census Data + Wrangling • Basic Visualization & Q1 &
\textbf{Lab 1 Start:} Census Exploration & Week 2 notes + Lab 1
progress \\
3 & Sep 22 & EDA • Visual Design Foundations & Q2 & Lab 1 continued &
Week 3 notes + Lab 1 progress \\
4 & Sep 29 & Spatial Operations • Neighborhood Indicators & Q3 &
\textbf{Lab 1 Due} + \textbf{Lab 2 Start} & Week 4 notes + Lab 1 final +
Lab 2 start \\
5 & Oct 6 & Linear Regression I • Making Predictions & Q4 & Lab 2
continued & Week 5 notes + Lab 2 progress \\
6 & Oct 13 & Linear Regression II • Model Evaluation & Q5 & \textbf{Lab
2 Due} + \textbf{Lab 3 Start} (Teams) & Week 6 notes + Lab 2 final + Lab
3 start \\
7 & Oct 20 & Spatial Autocorrelation • Intro to Spatial ML & Q5 &
\textbf{Lab 3 Continue} (Teams) & Week 7 notes + Lab 2 final + Lab 3
start \\
8 & Oct 27 & \textbf{House Prediction Presentations} + Count
Models-Predictive Policing & \textbf{Competition Presentations} &
\textbf{Lab 3 Due} & Week 8 notes + Lab 3 final + presentations \\
9 & Nov 3 & Logistic Regression I • Geographic Cross-Validation & Q6 & &
Week 9 notes \\
10 & Nov 10 & Logistic Regression II • Recidivism Case Study & Q7 &
\textbf{Lab 4 Start:} Parole Reform Analysis & Week 10 notes + Lab 4
start \\
11 & Nov 17 & Space-Time Modeling • Temporal Analysis & Q8 & \textbf{Lab
4 Due} + \textbf{Lab 5 Start:} Bike Share & Week 11 notes + Lab 4 final
+ Lab 5 start \\
12 & Nov 24 & Text Analysis + *k*-means clustering. Final Challenge
Introduced in Class & Q9 & Lab 5 continued & Week 12 notes + Lab 5
start \\
13 & Dec 1 & \textbf{Final Challenge Continued - Teams work in class.} &
Q10 & Lab 5 Due & Lab 5 final \\
14 & Dec 8 & \textbf{Final Challenge Presentations} & & & Final
Challenge Deliverables Due in 1 week. \\
\end{longtable}

\begin{center}\rule{0.5\linewidth}{0.5pt}\end{center}

\subsection{Academic Integrity \& AI
Policy}\label{academic-integrity-ai-policy}

\subsubsection{Core Principle}\label{core-principle}

All written work must be in your own words and demonstrate your
understanding of concepts.

\subsubsection{AI Tool Guidelines}\label{ai-tool-guidelines}

\begin{itemize}
\tightlist
\item
  \textbf{Permitted:} Using AI for debugging code, understanding error
  messages, understanding or decoding samples.
\item
  \textbf{Not Permitted:} Copying/pasting AI-generated text for
  assignments, having AI complete entire problems. Using AI to interpret
  your results or do your data analysis. Providing responses suggested
  by AI that you do not fully understand.
\item
  \textbf{Quiz Preparation:} Use AI to help understand concepts, but
  ensure you can explain ideas without assistance
\end{itemize}

\begin{center}\rule{0.5\linewidth}{0.5pt}\end{center}

\subsection{Additional FAQ
Information}\label{additional-faq-information}

\textbf{Late Assignments}: Please turn in your assignments on time. I do
very much understand that you have many other courses and I've done by
best to make the required work reasonable. However, the concepts in this
course build on each other and therefore assignments need to be turn in
on time.

\textbf{Revising \& Resubmitting Assignments}: I've minimized the
grading of assignments for reasons outlined above. You'll receive
general feedback on how to improve in future work, but as long as you
complete the assignment and continue to improve, you'll `pass' the
assignment. Therefore, there is no option to revise and resubmit for a
higher grade.

\textbf{Academic Integrity}:
\href{https://catalog.upenn.edu/pennbook/code-of-academic-integrity/}{\emph{Please
see the university policy on academic integrity}}. Cases of academic
dishonesty on assignments will result in a score of 0 on the assignment.

\textbf{Policy:} 24-hour response time goal. For coding issues, please
share your repository link or create a GitHub Issue for technical
problems.

\begin{center}\rule{0.5\linewidth}{0.5pt}\end{center}

\emph{This syllabus may be modified during the semester. Check Canvas
for the most current version.}




\end{document}
